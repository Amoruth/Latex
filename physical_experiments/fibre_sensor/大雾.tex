\documentclass[UTF8]{ctexart}
\usepackage{graphicx}

\title{\textbf{\songti\zihao{-2}光纤传感器}}
\author{\kaishu\zihao{-4}田丰睿    PB23071318}

\begin{document}

\maketitle  

\leftline{\textbf{\heiti\zihao{-5} 摘要:}}
\kaishu\zihao{-5} 
本次实验通过设计透射式横(纵)向光纤位移传感、反射式光纤位移传感和微湾光纤位移传感三个小实验,研究光纤传感。分别设计了透射式的横纵向位移、反射式纵向位移、微弯位移逐渐变化,观察接收到的光强的变化趋势。发现在透射式光纤位移实验中,当纵向位移逐渐增大时,光强逐渐减小;而当横向位移逐渐增大时,光强先增大再减小;在反射式光纤位移实验中,当纵向位移逐渐增大时,光强先是显著增大而后逐渐降低。其变化趋势满足反射式调制特性曲线;在微弯位移传感实验中,光强随微弯位移的增大而逐渐增大。\\
\vspace{1em}

\leftline{\textbf{\heiti\zihao{-5} 关键词:}\kaishu\zihao{-5} 光纤、传感器、位移}
\vspace{1em}

\CTEXsetup[format={\Large\bfseries}]{section}

\section{引言}
光纤传感器件具有体积小、重量轻、抗电磁干扰强、防腐性好、灵敏度高等优点;用于测量压力、应变、微小折射率变化、微振动、微位移等诸多领域。按传感原理可分为功能型和非功能型。光纤传感器可以探测的物理量很多,无论是探测哪种物理量,其工作原理都是用被测量的变化调制传输光光波的某一参数,使其随之变化,然后对已调制的光信号进行检测,从而得到被测量。将光纤和传感器结合一起的光纤传感器是将力热光电等物理基本概念和原理结合起来的综合实验。通过设计透射式横(纵)向光纤位移传感、反射式光纤位移传感和微湾光纤位移传感三个不同的小实验,可以研究在传播路径变化的情况下接收器接收到的光强的变化情况。
\vspace{1em}

\section{实验内容与设计}

\subsection{实验仪器}
\par
光纤输出 650nm 半导体激光器、光纤准直镜、反射式传感用光纤、透射式传感用光纤、电光晶体、光纤功率计、反射镜、控制电源。

\subsection{实验原理}
\noindent{\bfseries 1.透射式横(纵)向光纤位移传感}
\par
透射式光纤位移传感器是一种强度光纤传感,通过改变两透射多模光纤的光芯的相对位置(横向或者纵向),观测传输功率的变化,从而可以得到绘制功率随横向或纵向位移的关系曲线。
\par
透射式强度调制光纤传感原理如下图所示,调制处的光纤端面为平面,采用控制发射光纤不动,对接收光纤进行横向或纵向位移的方法,实现光纤被横向位移和纵向位移调制。
\begin{figure}[!h]
    \centering
    \includegraphics[width=0.5\linewidth]{光纤//figure/光纤位移传感器示意图.png}
    \caption{光纤位移传感器示意图}
    \label{img1}
\end{figure}
\vspace{1em}

\noindent{\bfseries 2.反射式光纤位移传感器}
\par
反射式光纤传感实验的光纤探头由两根光纤组成,一根用于发射光,一根用于接收反射回来的光,而另一段放置的是反射材料。由发射光纤发出的光照射到反射材料上,通过检测反射光的强度变化,能够缉拿测出反射体的位移。
\par
整个系统可在两个区域中工作,前沿工作区和后沿工作区(见下图理论曲线)。当在后沿区域中工作时,可以获得较宽的动态范围。
\begin{figure}[!h]
    \centering
    \includegraphics[width=0.5\linewidth]{光纤/figure/反射式调制特性曲线.png}
    \caption{反射式调制特性曲线}
\end{figure}
\vspace{1em}

\noindent{\bfseries 3.微弯光纤位移传感器}
\par
微弯型光纤传感器的原理结构如下图所示。当光纤发生弯曲时,由于其全反射条件被破环,纤芯中传播的某些模式光束进入包层,造成光纤中的能量损耗。
\begin{figure}[!h]
    \centering
    \includegraphics[width=0.5\linewidth]{光纤/figure/微弯位移传感器示意图.png}
    \caption{微弯位移传感器示意图}
\end{figure}
\par
在本次实验中,为了放大这种效应,将光纤夹持在一个周其波长为A的梳妆结构中。当梳妆结构(变形器)受力时,光纤的弯曲情况发生变化,因而纤芯中跑到包层中的光能(即损耗)也将发生变化。
\vspace{2em}

\subsection{实验过程}

\noindent{\bfseries 1.透射式横(纵)向光纤位移传感}
\par
根据\ref{img1}所示原理和实验室光具座搭建光路。
\par
纵向:当r取定时,改变z可以得到纵向位移传感特性。取功率计读数最大为初始位置,步长为 0.10mm 测量功率值。
\par
横向:当z取定时,改变r可以得到横向位移传感特性。取功率计读数最大为初始位置,左右移动 0.01mm 步长测量功率值。
\par
纵向和横向位移传感实验数据测。
\par
记录纵向和横向位移传感实验数据。
\vspace{1em}

\noindent{\bfseries 2.反射式光纤位移传感}
\par
根据反射式光纤位移传感实验原理搭建光路,自左向右以次为光纤光源(波长 650nm,功率 2mW)、功率计、反射镜、Y 型光纤(可四维调整)。
\par
逐渐将 Y 型光纤出射端和反射镜靠近(光纤输出端不要与反射镜接触),待不能再靠近时,调节 Y 型光纤的姿态(四维调整架的二维俯仰旋钮)使功率计示数最大。之后将 Y 型光纤靠近反射镜,从距离最近处开始测量,0.10mm 步长。
\par
记录反射位移传感实验数据。
\vspace{1em}

\noindent{\bfseries 3.微弯光纤位移传感}
\par
根据微弯型光纤位移传感器原理使用实验室提供的仪器搭建光路,自左向右以次为光纤光源(波长 650nm,功率 2mW,光源输出光纤与微弯光纤中间用光纤法兰连接),微弯光纤(变形器),干板夹和功率计。
\par
光源与功率计用微弯光纤连接起来,将微弯光纤夹持在变形器上,逐渐靠近两个变形器,同时观察功率计示数,待功率计示数开始变小或者纤微露红光时开始测量。以 0.05mm 作为步长进行多次测量。
\par
记录微弯光纤位移传感实验数据。
\vspace{1em}

%\subsection{注意事项}

\section{实验结果与讨论}

\subsection{结果和讨论}
%----下面写结果和讨论----

\subsubsection{实验结果}

\begin{table}[!h]
    \centering
    \begin{tabular}{|l|l|l|l|l|l|l|l|}
    \hline
    纵向位置(mm) & 14.45 & 14.55 & 14.65 & 14.75 & 14.85 & 14.95 & 15.05 \\ \hline
    功率(uW)   & 8.95  & 8.45  & 7.42  & 6.55  & 5.81  & 5.40  & 5.00  \\ \hline
    纵向位置(mm) & 15.15 & 15.25 & 15.35 & 15.45 & 15.55 & 15.65 & 15.75 \\ \hline
    功率(uW)   & 4.35  & 4.02  & 3.89  & 3.82  & 3.61  & 2.88  & 2.59  \\ \hline
    \end{tabular}
    \caption{透射式纵向位移}
    \label{tab:my_label}
\end{table}

\vspace{1em}

\begin{table}[!h]
    \centering
    \begin{tabular}{|l|l|l|l|l|l|l|l|l|l|}
    \hline
    横向位置(mm) & 14.28 & 14.27 & 14.26 & 14.25 & 14.24 & 14.23 & 14.22 & 14.21 & 14.20 \\ \hline
    功率(uW)   & 10.98 & 9.83  & 9.68  & 8.72  & 7.51  & 6.98  & 6.80  & 6.62  & 5.91  \\ \hline
    横向位置(mm) & 14.29 & 14.30 & 14.31 & 14.32 & 14.33 & 14.34 & 14.35 & 14.36 & 14.37 \\ \hline
    功率(uW)   & 8.33  & 6.32  & 5.89  & 5.53  & 5.16  & 4.96  & 4.12  & 3.15  & 2.60  \\ \hline
    \end{tabular}
    \caption{透射式横向位移}
\end{table}

\vspace{1em}

\begin{table}[!h]
    \centering
    \begin{tabular}{|l|l|l|l|l|l|l|l|l|l|}
    \hline
    纵向位置(mm) & 11.87 & 11.97 & 12.07 & 12.17 & 12.27 & 12.37 & 12.47 & 12.57 & 12.67 \\ \hline
    功率(uW)   & 32.88 & 66.87 & 104.3 & 142.7 & 171.5 & 190.3 & 200.2 & 206.6 & 210.1 \\ \hline
    纵向位置(mm) & 12.77 & 12.87 & 12.97 & 1307  & 13.17 & 13.27 & 13.37 & 13.47 & 13.57 \\ \hline
    功率(uW)   & 210.0 & 211.3 & 209.6 & 208.1 & 203.3 & 198.5 & 192.7 & 186.5 & 179.8 \\ \hline
    \end{tabular}
    \caption{反射式纵向位移}
\end{table}

\vspace{1em}

\begin{table}[!h]
    \centering
    \begin{tabular}{|l|l|l|l|l|l|l|l|l|l|}
    \hline
    微弯位移(mm) & 18.25 & 18.20 & 18.15 & 18.10 & 18.05 & 18.00 & 17.95 & 17.90 & 17.85 \\ \hline
    功率(uW)   & 6.78  & 6.46  & 6.12  & 5.68  & 5.37  & 5.00  & 4.76  & 4.52  & 4.30  \\ \hline
    微弯位移(mm) & 17.80 & 17.75 & 17.70 & 17.65 & 17.60 & 17.55 & 17.50 & 17.45 & 17.40 \\ \hline
    功率(uW)   & 4.15  & 3.97  & 3.87  & 3.71  & 3.56  & 3.44  & 3.29  & 3.15  & 3.00  \\ \hline
    \end{tabular}
    \caption{微弯位移}
\end{table}

\vspace{1em}

\subsubsection{讨论}
根据所测得的各组数据,得到了如下的几张散点图表。


\vspace{1em}
\begin{figure}[!h]
    \centering
    \includegraphics[width=0.5\linewidth]{光纤//figure/纵向散点.png}
    \caption{透射式纵向位移}
\end{figure}
\vspace{1em}

\begin{figure}[!h]
    \centering
    \includegraphics[width=0.5\linewidth]{光纤//figure/横向散点.png}
    \caption{透射式横向位移}
\end{figure}
\vspace{1em}

从图4和图5两图可以看出,当纵向位移逐渐增大时,光强逐渐减小;而当横向位移逐渐增大时,光强先增大再减小。

\begin{figure}[!h]
    \centering
    \includegraphics[width=0.5\linewidth]{光纤//figure/反射散点.png}
    \caption{反射式纵向位移}
\end{figure}
\vspace{1em}

从图6可以看见,在反射位移传感实验中,当纵向位移逐渐增大时,光强先是显著增大而后逐渐降低。其变化趋势满足反射式调制特性曲线。

\begin{figure}[!h]
    \centering
    \includegraphics[width=0.5\linewidth]{光纤//figure/微弯位移.png}
    \caption{微弯位移}
\end{figure}
\vspace{1em}

从图7所示散点图可以看出,在微弯位移传感实验中,光强随微弯位移的增大而逐渐增大。


\section{结论与反思}

\subsection{结论}
本次实验通过在不同情况下测量光强发现在透射式光纤位移实验中,当纵向位移逐渐增大时,光强逐渐减小;而当横向位移逐渐增大时,光强先增大再减小;在反射式光纤位移实验中,当纵向位移逐渐增大时,光强先是显著增大而后逐渐降低。其变化趋势满足反射式调制特性曲线;在微弯位移传感实验中,光强随微弯位移的增大而逐渐增大。
\subsection{反思}
在实验过程中,发现在使用功率计测量接收光纤接收的光束的强度时,光强始终在不停的变化。甚至在不同时间,各实验装置处于同一空间位置的情况下,光强会有较大的差距。通过实验之后的查阅资料,发现这是由输出激光发出的光源不稳定、功率计过于灵敏等一众原因造成的正常现象,在实验允许范围之内。

\vspace{2em}

\leftline{\textbf{\songti\zihao{-4}参考文献}}\vspace{1em}
[1] 光纤传感器 . 实验讲义 . 2024

\vspace{2em}

\leftline{\textbf{\songti\zihao{-4}实验数据}}

\begin{figure*}
    \centering
    \includegraphics[width=\linewidth]{光纤/figure/光纤数据.jpg}
\end{figure*}

\end{document}

